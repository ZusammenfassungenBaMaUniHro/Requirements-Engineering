\section{Designing Participatively - Teilnehmenderweise Entwurf}
Arten von Teilnahme:
\begin{itemize}
	\item Konsultativ/beratend
	\item Repräsentativ
	\item Konsensus - zustimmend
\end{itemize}
Zwei Ansätze zur Arbeitsgestaltung:
\begin{itemize}
	\item \textbf{Job Bereicherung (Enrichment)}: Verbessern der Beziehung von Arbeiter und seiner Arbeit
	\item \textbf{soziotechnische Methode}
	\begin{itemize}
		\item verbessern der Beziehung zw Gruppe und ihrer Arbeit
		\item logische Analyse der technischen Komponenten des Arbeitssystems
		\item Identify Systemabweichungen
	\end{itemize}
\end{itemize}

\textbf{First Task of Design Group: Identification of Work System Problems}
\begin{itemize}
	\item Daten über Probleme, die die Effizienz der Gesamtarbeit beeinträchtigen
	\item Daten über Bedürfnisse für Jobzufriedenheit und über Dinge, die das aktuelle System zusätzlich dafür benötigt
	\item Daten wie das neue System die Effizienz und Zufriedenheit verbessern kann
\end{itemize}

\textbf{Problem solving processes}
\begin{enumerate}
	\item Problem verstehen durch Analyse von vielen, vielen Daten
	\item große Datenmengen durcharbeiten und aussortieren/Auswahl was relevant ist um das Problem zu lösen
	\item kleinere Datenmenge für Problemlösung nutzen
\end{enumerate}

\textbf{Strategies for Design process}
\begin{enumerate}
	\item 'Ideal' Systemansatz
	\item Baukastenansatz
\end{enumerate}

Zentrale Aufgabe des RE ist die Einbettung des neuen Systems in die bestehende Arbeitssituation/Anwendungskontext.\\
Relevante Zielstellungen:
\begin{itemize}
	\item Unternehmenssicht\\
	mehr Effizienz in Ablauf- und Aufbauorganisation
	
	\item Arbeitswissenschaftlich\\
	Steigerung der Arbeitseffektivität\\
	Optimierung der psychophysischen Beanspruchung der Arbeiter\\
	Stabilisierung der Gesundheit
	
	\item Softwaretechnische Sicht\\
	Entwicklung von aufgabenangemessenen Softwaresystemen
\end{itemize}

\textbf{Humane Factors/Ergonomics (HFE)}\\
Wissenschaftliche Disziplin, die sich mit der Interaktion zw Menschen und anderen Elementen eines Systems und dem Beruf, der Theorie, Daten und Methoden befasst mit dem Ziel das Wohlbefinden des Menschen und die gesamte Systemperformance zu verbessern.\\
zugehörige Ansätze
\begin{itemize}
	\item Ganzeinheitlicher Ansatz: Betrachte System und Menschen
	\item entwurfsgetrieben: Entwurfsmethoden beinhalten teilnehmende Ansätze -> Einbeziehen der Stakeholder
	\item Fokus auf zwei verbundene Ergebnisse: Wohlbefinden und Performance\\
	geringes Wohlbefinden verschlechtert die Performance und vice versa 
\end{itemize}
