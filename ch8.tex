\section{Product Lines and Feature Models}
Product Line = Serie von verschiedenen Produkten, die eine Gruppe formen und vom selben Hersteller stammen\\
Feature Model = Formalisierung von R; abstrahiert von konkreter Implementierung\\
Software Product Line = Menge von Ausprägungen eines Softwareprodukts, die auf derselben Plattform basieren; Jede Ausprägung besitzt zusätzliche, unterschiedliche Features die bestimmte Märkte/Kunden bedienen (Bsp: Windows 7 Produktlinie -> unterschiedliche Versionen)\\
\\
\textbf{Feature}\\
\begin{itemize}
	\item vom Benutzer sichtbare Aspekte/Charakteristiken des Systems
	\item Systemeigenschaft, die relevant für Stakeholder ist
	\item logische Einheit von Verhalten, das durch eine Menge von funktionalen und qualitativen R spezifiziert ist
	\item Feature f = (W; R; S)m mit f erfüllt Requirement R; W sind die Annahmen über das Umfeld; S = Spezifikation von f
	\item Stück einer Programmfunktionalität
\end{itemize}

\textbf{Feature Diagram}\\
Baum von Features; Wurzelknoten = Konzeptgedanke\\
Knoten = verbindliches oder optionales Feature\\
\textbf{Feature Model} = Feature Diagram + zusätzliche Daten, wie Featurebeschreibung; Prioriäten; Stakeholder;\\
helfen R von Produktlinien zu erkennen
\\
\textbf{Feature Interaction}\\
Interaktion zwischen Features zeigt sich nur, wenn diese zusammengesetzt werden und nicht individuell laufen; Beispiel:\\
Telekomm. Call Waiting and Call Forwarding; beide Features können nicht in Verbindung aktiviert werden, denn wenn eines eine Priorität besitzt, wird das anderes disabled\\
\\
\textbf{Feature Oriented Software Development:}\\
Feature Modeling; Feature Interaction und Feature Implementation sind Bestandteile\\
Features stehen an erster Stelle und bilden Grundbaustein für Entwicklung\\
Ablauf
\begin{enumerate}
	\item Domain Analysis
	\item Domain Design and Specification
	\item Domain Implementation
	\item Product Conifig and Generation
\end{enumerate}