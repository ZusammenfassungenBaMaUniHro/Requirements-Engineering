\section{Scenario-Based Analysis and Design}
Fokus auf Beschreibung wie Personen das System nutzen um eine Aufgabe zu erledigen.\\
Organisationen können sich aus zwei Sichten betrachten:
\begin{table}[h]
	\centering
	\begin{tabular}{|p{20em}|p{20em}|}
		\hline
		\textbf{Explizite Ansicht} & \textbf{Implizite Ansicht}\\
		\hline
		Aufgaben & Wissen\\
		Hierarchieposition	& Kontaktnetzwerk\\
		Workflow 	& Arbeitspraktiken\\
		Teams	& Communities\\
		\hline
	\end{tabular}
\end{table}
Klassifikation von Szenarios
\begin{itemize}
	\item Problemszenarios als Oberpunkt; Unterteilung in:
	\item Aktivitätenszenarios: typische oder kritische Dienste, die von Stakeholder erwartet werden
	\item Informationsszenarios: Daten die zw Nutzer und System ausgetauscht werden
	\item Interaktionsszenarios: Interaktion zw Nutzer und System
\end{itemize}


Übersicht der Analyse
\begin{enumerate}
	\item Root Concept of the system to be developed\\
	intiale Analyse der R; Auflistung der Schlüsselaspekte der Vision; dient als Ausgangspunkt für weitere Analysen\\
	Dient zur Entwicklung von Scenarios
	\item Field studies: Arbeitsplatz beobachten; Interviews; Artefakte bestimmen; Beziehungen zwischen Akteuren
	\item Summaries:
	\item Bestimmen von
		\begin{itemize}
			\item Problemszenarios: Darstellung des Ergebnisses der Field Studies -> Schreiben von Szenarios\\
			Ziel: Vorstellen von Themen und Beziehungen für die weitere Entwicklung
			\item Analyse von Ansprüchen (Claims): finden von Kernfeatures/Artefakten\\
			Ziel: Themen und implizite Beziehungen explizit herausarbeiten und für Diskussion nutzbar machen
		\end{itemize}
	eng miteinander verdrahtet
\end{enumerate}

\textbf{Elemente eines Szenarios}
\begin{itemize}
	\item Actor
	\item Umgebung
	\item Ziele
	\item Pläne
	\item Events
	\item Aktionen
\end{itemize}

\textbf{Zusammenfassung}
\begin{itemize}
	\item \textbf{Szenario}: Beschreibt Informationen über Akteure, ihre Annahmen über das Umfeld, ihre Ziele, Motive, Aktionen... etc\\
	Darstellung in verschiedenen Formen und Medien\\
	Informell, semi-formal, formale Notationen mgl
	\item \textbf{Claim/Anspruch}: Beschreibung eines Artefakts, das in Scenario vorkommt und Einfluss auf Akteure hat\\
	stellt implizite Beziehungen expliziter dar.
	\item Handlungsorientiert: Analyse der aktuellen Praxis und daraus neue Aktivitäten entwickeln
	\item erleichtern Nachdenken über komplexe Systeme
\end{itemize}


\section{Non-Functional Requirements Framework}
NFRs treiben den ganzen Entwurfsprozess voran bzw als zentraler Bestandteil des Prozesses.\\
NFRs stellen die Rechtfertigung für Design Entscheidungen und Bedingungen dar wie geforderte Funktionalitäten umgesetzt werden. Behandelt als \textbf{Softgoals}\\
\begin{itemize}
	\item Ziel, das nicht zu 100\%, aber gut genug erfüllt werden muss
	\item 1 Softgoal erfüllt oder lehnt/verweigert anderes softgoal ab.
\end{itemize}
Bestandteile/Ablauf des Frameworks
\begin{itemize}
	\item Erhalten von Wissen über: Anwendungsdomain; functional R für Teile des Systems; 
	\item Identifizieren und Zerlegen von einzelnen NFRs
	\item Identify Entwurfsalternativen für NFRs
	\item Auswerten/Behandeln von Trade Off; Prioritäten; Mehrdeutigkeiten
	\item Auswahl von Entwurfsalternativen
	\item Entscheidungen mit Design Rationale begründen
	\item Auswertung der Entscheidungen und deren Einfluss
\end{itemize}