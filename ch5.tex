\section{Z-Notation}
\begin{itemize}
	\item basiert auf Mengenlehre und Prädikatenlogik 1. Stufe
	\item beschreibt Zustände (durch Menge, Objekte, Prädikate) und Übergänge (durch Beschreibung des Zustands vor und nach Übergang) des Systems
	\item Schema beschreibt Zustand bzw. Übergang
	\item strukturiertes Wissen über Domäne wird als Ausgangspunkt zum Modellieren benutzt - \textit{Domäne} = exist. und aufgestellte Systeme die ähnliche Ziele haben\\
	Aufgestelltes System = System (Entitäten die miteinander interagieren um ein Domainziel zu erreichen) + Umwelt(Elemente, die Input geben und auf Output reagieren)
	\item \textbf{Modelling the domain}
	\begin{itemize}
		\item als Graph; V = Aussagen über zu erreichende Ziele; formale Spezi, das Verhalten des Systems beschreibt; Erklärungen (natürliche Sprache)
		\item E = Beziehungen zw Zielen verschiedener $v \in V$; 4 Arten:\\
		is\_supported\_by; is\_undermined\_by(untergraben; x wird ruch y gestört); may\_be\_specialised\_by; must\_be\_considered\_before(geprüft)
		\item iterativer prozess; Aufstellen von high und low level goals:\\
		high-level: Knoten haben selten Fragmente formaler Spezi\\
		low-level: Knoten haben Fragmente formaler Spezi
		\item Basic System Node: sollte die Basisfunktionalität des Systems beinhalten; mit Stakeholder abgeglichen werden; Ausgangspunkt für weitere Spezifikationen
	\end{itemize}
\end{itemize}

\begin{table}[!h]
	\centering
	\begin{tabular}{|p{20em}|p{20em}|}
		\hline
		Pros	& Cons\\
		\hline
		Formal, Strukturiert	& R des Kunden formalisieren nicht trivial\\
		besseres Verständnis der R durch Domain Model	& iterative Entwicklung kann inkonsistente Spezi hervorrufen\\
		\hline
	\end{tabular}
\end{table}