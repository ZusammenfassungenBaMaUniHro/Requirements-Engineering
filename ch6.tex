\section{Design Rationale (DR)}
Entwicklungsansatz das die expliziten Gründe, die zum Produkt führten, und deren Darstellung beinhaltet. Quasi welche konkrete Entscheidungen haben zum Produkt geführt?\\
Oftmals nicht gut dokumentiert, da Entwicklungsprozess sehr komplex: bestehend aus Analyse (Problem in Subprobleme teilen) und Synthese (Teillösungen für Subprobleme zu Gesamtlösung zusammenfügen).

\begin{itemize}
	\item Design ist fortlaufender Prozess, mit Betrachtung und Entscheidung von Alternativen; da Teilprobleme oft nicht unabhängig voneinander
	\item \textbf{Wicked} und Tamed Problems: schwere und klar abgegrenzte Probleme
	\begin{itemize}
		\item jeder Lösungsversuch ändert das Problemverständnis
		\item schwer bestimmbar WANN Problem gelöst, da Problem selbst nicht genau definiert werden kann
		\item Stakeholder müssen sich auf hinreichend gute Lösung einigen
		\item $\ldots$
	\end{itemize}
	
\end{itemize}

\begin{table}[h]
	\begin{tabular}{|p{20em}|p{20em}|}
		\hline
		Pros	& cons\\
		\hline
		Aufzeichnen der Entscheidungen & \\
		Kommunikation innerhalb des Projekts & \\
		Integration verschiedener Stakeholder Perspektiven und Interessen & \\
		Generieren von Ideen & \\
		Anzeigen ungelöster/wiederkehrender Probleme & \\
		
	\end{tabular}
\end{table}

\textbf{Voraussetzung}\\
Stakeholder sind bereit für Argumentationsprozess und rationale Entscheidungen zu treffen. Rationale Entscheidungen verkörpern oft fehlendes Wissen.

\subsection{Design Space Analysis and QOC Notation - ein DR Ansatz}
DSA: Verstehen eines Software Produkts; Systematische Exploration und Auswertung des Entwurfsraum mit Ziel sich für eine Alternative als Lösung zu entscheiden.\\
Herausforderung für Devs - Aufstellen der QOC Notation zur Unterstützung des DSA\\
\begin{itemize}
	\item Ask \textbf{Q}uestions: zur Strukturierung/Vergleich von Optionen und Erstellen neuer Optionen; Option = Lösung/Antwort auf Problem/Frage
	\item create \textbf{O}ptions; Beispiel konkrete Eigenschaft des Produkts
	\item List \textbf{C}riteria zur Bewertung von Optionen und schlussendlich auch zu deren Auswahl
	Characteristics:
	\begin{itemize}
		\item Messen der 'Eigenschaften' eines Kriteriums
		\item muss beurteilbar sein
		\item Kann sich auf (allgemeinere) Criteria beziehen
		\item muss unabhängig von anderen Criteria in derselben Hierarchieebene sein 
	\end{itemize}
	\item Arguments: unterstützen die Bewertung der Optionen (Bsp: Argument ist Grund dafür das Option 2 besser als Option 1 bewertet wurde)
\end{itemize}
Questions and Options erschaffen eine hierarchische Struktur - den Entwurfsraum -> liefern \textbf{QOC Diagram}\\
- während den Meeting zur Diskussionslenkung benutzbar, indem Teilnehmer an Ideen der DSA orientieren\\
- nach Meeting zur strukturierten Protokollierung\\
QOC dient als verständliche Darstellung für verschiedene Stakeholder\\
\textbf{Grobe QOC:} Ideen des QOC Ansatzes nutzen für die fortlaufende Diskussion\\
\textbf{Gründliche QIC} : Protokollierung; Transkribieren; Analyse des Protokolls; QOC Diagram\\
\\
\textbf{Fazit:}\\
Notation einfach genug um für alle verständlich zu sein; Flexibel genug um verschiedene Perspektiven zu repräsentieren; genau genug um Annahmen anderer zu dokumentieren\\
QOC Diagramme nutzen für Probleme, die keine beste oder offensichtliche Lösung haben
